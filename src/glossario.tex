% Esse é o arquivo de variáveis do estilo da SOCIESC-IST
% Quando definir e alterar as suas variáveis, observe atentamente a caixa
%  em que o texto está escrito.

\acro{API}			[API]{\textit{Application Programming Interface}}
\acro{browser}		[browser]{\textit{navegador web}}
\acro{bug}			[bug]{\textit{funcionamento comum de um software, ou ainda, falha
lógica de um programa}}
\acro{CGI}			[CGI]{\textit{Common Gateway Interface }}
\acro{CSS}			[CSS]{\textit{Folhas de Estilo em Cascata}}
\acro{DBMS}			[DBMS]{\textit{Database Management System}}
\acro{E/S}			[E/S]{\textit{Entrada/Saída}}
\acro{Excel}		[Excel]{\textit{aplicativo desenvolvido pela Microsoft para
criação de planilhas eletrônicas}}
\acro{GUI}			[GUI]{\textit{Graphics User Interface}}
\acro{HTML}			[HTML]{\textit{Linguagem de Marcação de Hipertexto}}
\acro{HTTP}			[HTTP]{\textit{Hypertext Transfer Protocol}}
\acro{I/O}			[I/O]{\textit{Input/Output}}
\acro{JS}			[JS]{\textit{Javascript}}
\acro{JSON}			[JSON]{\textit{JavaScript Object Notation}}
\acro{LESS}			[LESS]{\textit{Leaner CSS}}
\acro{MySQL}		[MySQL]{\textit{sistema gerenciador de banco de dados}}
\acro{Nginx}		[Nginx]{\textit{servidor web de alta performance orientado a
eventos}}
\acro{OOP}			[OOP]{\textit{Object Oriented Programming}}
\acro{ORM}			[ORM]{\textit{Object Relational Mapper}}
\acro{OSCON}		[OSCON]{\textit{Open Source Convention}}
\acro{open source}	[open source]{\textit{código aberto}}
\acro{PDO}			[PDO]{\textit{‎PHP Data Objects}}
\acro{PHP}			[PHP]{\textit{Pré-Processador de Hipertexto}}
\acro{PHP/FI}		[PHP/FI]{\textit{Personal Home Page Tools/Forms Interpreter}}
\acro{PHP-GTK}		[PHP-GTK]{\textit{extensão para a linguagem de programação PHP
que implementa o binding da linguagem para o GTK+}}
\acro{Pearson VUE}	[Pearson VUE]{\textit{serviço de teste eletrônico da Pearson Education}}
\acro{POO}			[POO]{\textit{Programação Orientada a Objetos}}
\acro{PostgreSQL}	[PostgreSQL]{\textit{sistema gerenciador de banco de dados}}
\acro{RDBMS}		[RDBMS]{\textit{Relational Database Management System}}
\acro{REST}			[REST]{\textit{REpresentational State Transfer}}
\acro{RoR}			[RoR]{\textit{Ruby on Rails}}
\acro{SASS}			[SASS]{\textit{Syntactically Awesome Style Sheets}}
\acro{SF2}			[SF2]{\textit{Symfony 2}}
\acro{SGBD}			[SGBD]{\textit{Sistema Gerenciador de Banco de Dados}}
\acro{SGBDR}		[SGBDR]{\textit{Sistema Gerenciador de Banco de Dados
Relacionais}}
\acro{SOAP}			[SOAP]{\textit{Simple Object Access Protocol}}
\acro{SQL}			[SQL]{\textit{Structured Query Language}}
\acro{TCC}			[TCC]{\textit{Trabalho de Conclusão de Curso}}
\acro{TI}			[TI]{\textit{Tecnologia da Informação}}
\acro{Twitter Bootstrap}[Twitter Bootstrap]{\textit{biblioteca de
componentes responsivos para a web}}
\acro{XML}			[XML]{\textit{eXtensible Markup Language}}
\acro{XSS}			[XSS]{\textit{Cross-Site Scripting}}
\acro{YAML}			[YAML]{\textit{YAML Ain't Markup Language}}
\acro{ZCE}			[ZCE]{\textit{Zend Certified Engineer}}
\acro{ZCPE}			[ZCPE]{\textit{Zend Certified PHP Engineer}}
\acro{ZFC}			[ZFC]{\textit{Zend Framework Certification}}
\acro{Zend Framework}[Zend Framework]{\textit{framework PHP}}
\acro{Zend Server}	[Zend Server]{\textit{servidor web}}
\acro{Zend Studio}	[Zend Studio]{\textit{IDE baseada no projeto eclipse voltada
ao desenvolvimento utilizando a linguagem PHP}}
\acro{Zend}			[Zend]{\textit{Empresa norte-americana fabricante de software}}
\acro{ZF2}			[ZF2]{\textit{Zend Framework 2}}