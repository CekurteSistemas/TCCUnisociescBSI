\section{FRAMEWORK}

Os \textit{frameworks} surgiram com o objetivo de facilitar o desenvolvimento de
software, pois utilizando-o, programadores focam os seus esforços no que é
realmente importante, as regras de negócio do sistema.

Conforme afirma \citeonline{padroesDeProjetosSolucoesReutilizaveis}, os
\textit{frameworks} ditam a arquitetura de uma aplicação, eles cooperam afim de fornecer
uma abstração de tarefas que são comuns entre diferentes sistemas, sendo que,
através da combinação de padrões de projetos é possível que ele possa ser
reutilizado em diferentes projetos de maneira transparente permitindo que um
projetista possa se concentar nos aspectos específicos da aplicação.

Para \citeonline{frameworksParaDesenvolvimentoEmPHP}, um dos motivos para o
emprego de \textit{frameworks} nos projetos de software está no fato de minimizar as
tarefas repetitivas, permitindo que uma equipe de desenvolvimento entregue
soluções de maneira mais rápida e com uma qualidade superior.

Os \textit{frameworks} disponíveis no mercado permitem que equipes de
desenvolvimento de software, utilizem um padrão de arquitetura da informação
evitando reiventar a roda a cada novo projeto, sendo assim, para
\citeonline{zendFrameworkComponentesPoderososParaPHP} o principal objetivo para
a adoção de \textit{frameworks} está no fato de que ele simplifica o
desenvolvimento de software, bem como promove as melhores práticas definidas por
um conjunto de pessoas, permitindo desenvolver aplicações mais seguras, modernas e
confiáveis.

A seguir, serão apresentados os dois \textit{frameworks} que são referência no
mercado quando trabalha-se com a linguagem \acs{PHP}, sendo ambos de código
livre, são eles: o \ac{SF2} e o \ac{ZF2}.

\subsection{Symfony 2}

O \textit{Symfony} foi desenvolvido por uma empresa francesa chamada
\textit{Sensio Labs}, sendo \textit{Fabier Potencier} um dos principais nomes
quando menciona-se este projeto, sendo que, inicialmente foi utilizado apenas em
projetos privados, tendo sua primeira versão de código aberto publicada em 2005 \cite{buildingPHPApplicationsWithSymfonyCakePHPAndZendFramework}.

Conforme define
\citeonline{buildingPHPApplicationsWithSymfonyCakePHPAndZendFramework}, o
\textit{Symfony} foi desenvolvido com base no \textit{Mojavi MVC framework},
com fortes influências do \ac{RoR}, além disto, ele integrou o \textit{Propel} e
posteriormente o \textit{Doctrine} ambos considerados um \ac{ORM} além do
\ac{YAML} que é um padrão de serialização de dados.

O \ac{SF2} disponível desde 2010, incluí vários novos recursos com um aumento
considerável de desempenho, atualmente é um dos principais frameworks web
contando com uma excelente documentação e uma comunidade ativa relativamente
grande \cite{buildingPHPApplicationsWithSymfonyCakePHPAndZendFramework}.

\subsection{Zend 2}

O \ac{ZF2} é a última atualização do já conhecido \textit{Zend
Framework}, foi e continua sendo desenvolvido pela \acs{Zend}, sendo que, nesta
nova versão, o processo para criação de aplicações web complexas está ainda mais
simplificado tudo isto graças a utilização de componentes fracamente acoplados.

Sendo assim, conforme descrito por
\citeonline{zendFramework2ByExampleBeginnersGuide}, o framework \acs{ZF2}
fornece uma estrutura altamente robusta e escalável para o desenvolvimento de 
aplicações web. Do ponto de vista de
\citeonline{criandoAplicacoesPHPComZendEDojoPadroesEReusoComFrameworks}, o 
\textit{Zend Framework} encapsula a experiência do desenvolvimento de aplicações
que utilizam a linguagem \acs{PHP} graças a sua biblioteca de componentes 
reutilizáveis.

Conforme descrito por \citeonline{zendFramework2ByExampleBeginnersGuide}, a
seguir, serão apresentadas as principais novidades desta nova versão do 
\textit{framework} desenvolvido pela \acs{Zend} comparando-o com a primeira
\textit{release}, são elas:

\begin{alineas}
    \item suporte aos recursos de namespaces e closures do \acs{PHP} 5.3;
    \item desenvolvimento de aplicações com arquitetura modular;
    \item suporte a gerenciador de eventos;
    \item suporte a injeção de dependência.
\end{alineas}
