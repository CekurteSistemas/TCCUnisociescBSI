\subsection{Polimorfismo}

Polimorfismo é a capacidade que dois ou mais objetos de uma classe-filha tem  de
responder a mensagens de diferentes formas
\cite{php5ConceitosProgramacaoEIntegracaoComBancoDeDados}.

Ou seja, polimorfismo é a possibilidade que um objeto tem de alterar o
comportamento de um objeto com base em uma classe especialista, sendo assim,
são métodos que fornecem resultados distintos  de acordo com a subclasse
\cite{php5ConceitosProgramacaoEIntegracaoComBancoDeDados}. Desta forma quem
chama o método não precisa distingui-lo.

A seguir, será apresentado a atribuição deste conceito no processo de aceleração
de um veículo. Por conta disto, imagine o exemplo a seguir: dentre os comandos
disponíveis em um veículo, tem-se o acelerador que fornece uma interface
encapsulando a forma que compreende em como as coisas funcionam. Agora vamos
supor que vamos acelerar dois veículos diferentes, o primeiro veículo trata-se
de um carro popular com motor 1.0, enquanto que, o outro veículo é um
 esportivo, e por conta disto, estamos falando de um carro com motor 2.0. Então,
ambos os veículos possuem a mesma interface de comunicação: o pedal acelerador,
que permite ao condutor se locomover de maneira mas rápida. Note, que quando o
condutor acelerar o carro popular este deverá ter aceleração mais lenta se
comparado ao esportivo, na prática é disto que se trata o polimorfismo.

Voltando-se para o paradigma da orientação a objetos seria possível existir uma
classe generalista chamada \textit{Carro} que seria responsável em definir um
meio padronizado para realizar a aceleração de um veículo. E, com base nesta
classe, criar duas outras classes especialistas, sendo que, poderiam ser
nomeadas como: \textit{CarroPopular} e \textit{CarroEsportivo}.

A seguir será apresentada a implementação destes conceitos
utilizando a linguagem de programação PHP.

\textit{Imagine o exemplo a seguir onde existe uma classe carro que possuí um
método acelerar e que existam duas outras classes que extendem a classe carro, uma
delas é a Lamborghini a outra é “Ford Fiesta”, ambos os veículos
aceleram, entretanto a lamborgini deverá ter uma aceleração superior, sendo assim irá
reescrever o método acelerar de acordo com as suas características, os objetos
que chamarem esta classe não saberam como acontece mas saberam que eles
realmente  aceleram de maneira correspondente ao tipo de veículo que está sendo
testado.}