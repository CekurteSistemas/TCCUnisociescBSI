\section{BANCO DE DADOS}

Antes de abordar o banco de dados \acs{MySQL} e o \acs{PostgreSQL} - ambos
bancos livres bastante populares no mercado - é necessário compreender
conceitualmente o que é um banco ou base de dados.
 
Segundo \citeonline{theDefinitiveGuideToMySQL5} um banco de dados pode ser uma
lista de registros que são manipulados por um programa de computador, como por
exemplos o \acs{Excel}, ou ainda, pode ser também os arquivos de armazenamento
de uma empresa de telecomunicações referente as várias chamadas que ocorreram
diariamente, além disto, alguns bancos de dados são utilizados por apenas um
usuário, enquanto que outros são acessados por vários usuários
simultaneamente, por conseguinte, uma base pode ocupar poucos ou muitos
kilobytes do dispositivo de armazenamento.

Sendo assim, a palavra \textit{banco de dados} é utilizada para referenciar os
dados reais, o software gerenciador do banco (como por exemplo: \acs{MySQL} e 
\acs{PostgreSQL}), um cliente de conexão ao banco (são exemplos: um script
\acs{PHP} ou um programa escrito em C++), sendo assim, é comum que quando as 
pessoas falem sobre o assunto gerem confusão \cite{theDefinitiveGuideToMySQL5}.

Portanto, conforme afirma \citeonline{phpMySQLForDummies}, o termo banco de
dados, do inglês \textit{database}, se refere a um arquivo ou grupo de arquivos contendo dados, 
sendo que, esses dados são acessados por programadas chamados de \ac{DBMS}.
Quase todos os \acs{DBMS} são \ac{RDBMS}, nos quais
os dados são organizados e armazenados em tabelas relacionais.

Os termos \acs{DBMS} e \acs{RDBMS} também são definidos respectivamente como
\ac{SGBD} e \ac{SGBDR}.

\subsection{MySQL}

De acordo com \citeonline{integrandoPHP5ComMySQL}, \acs{MySQL} é um
\acs{RDBMS},  que utiliza a linguagem \ac{SQL} para manipular os seus registros, 
sendo que, é altamente utilizado em aplicacções para a internet e, por
conseguinte,  seu código fonte é aberto, tendo destaque em características
como: velocidade, escalabilidade e confiabilidade, por conta disto, é adotado
pelo departamento de \ac{TI}, desenvolvedores e vendedores de software.

\subsection{PostgreSQL}

Assim como o \acs{MySQL}, o \acs{PostgreSQL} é um \acs{RDBMS} também de
código livre e seu antepassado ficou conhecido como Ingres, sendo que, surgiu na
universidade da Califórnia, em Berkeley
\cite{postgreSQLIntroductionAndConcepts}.

Por conseguinte, possuí alguns recursos de bancos empresariais, tais como: a
possibilidade de criar funções agregadas e também a replicação de
\textit{streaming},  certamente, estes recursos raramente são encontrados em 
plataformas livres, mas, são comumente encontrados em bancos comerciais, como:
\textit{SQL Server} e \textit{IBM DB2} e, além destas vantagens, ele pode 
superar bases comerciais  em algumas cargas de trabalho 
\cite{postgreSQLUpAndRunning}.
