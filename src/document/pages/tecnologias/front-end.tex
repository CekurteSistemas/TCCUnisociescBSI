\section{DESENVOLVIMENTO FRONT-END}

Geralmente quando desenvolve-se para a web, existem profissionais especialistas
em duas áreas da construção de uma aplicação, são eles: o \textit{frontend} e o
\textit{backend} \cite{artigoAvaliacaoEReducaoDoTempoDeRespostaDeSistemasWeb}.

Segundo \citeonline{artigoAvaliacaoEReducaoDoTempoDeRespostaDeSistemasWeb}, é no
\textit{frontend} que tecnologias como: a \ac{HTML}, as \ac{CSS}, o \ac{JS}
e o conteúdo multimídia são desenvolvidos. Por este motivo, neste capítulo estas
tecnologias serão abordadas.

\subsection{HTML}

Segundo \citeonline{htmlCSSTheGoodParts} na construção de uma aplicação web ou
um website, uma das tarefas mais importantes é a criação de links entre os
documentos, possibilitando assim a navegação, o \acs{HTML} permite criar
estes documentos, descrevendo o conteúdo que os usuários acessam ao explorar a
web.

Ao acessar um website, o \acs{HTML} informa ao navegador como o seu documento
foi estruturado: qual o título do documento, se existem parágrafos, se existe
algum texto enfatizado, quais são os botões de navegação. Sendo assim, o
navegador renderiza o documento, ou seja, faz a intepretação dos
comandos e exibe para o usuário o resultado final, que neste caso, é a página
web \cite{headFirstHTMLWithCSSAndXHTML}.

Entretanto, o \acs{HTML} não trata a forma como os elementos estão dispostos em
tela e para isto existe uma outra tecnologia responsável por manipular os
elementos visuais, trabalhando de forma integrada ao \acs{HTML}:
o \acs{CSS}, que será apresentado a seguir.

\subsection{CSS}

O \ac{CSS}, do inglês \textit{Cascading Style Sheets}, é uma ferramenta
que \textit{webdesigners} e também desenvolvedores utilizam em conjunto com o
\ac{HTML} para a construção de websites, onde, o \acs{CSS} proporciona que \ac{browser}
controle os aspectos visuais da página, como o posicionamento de elementos,
estilos de texto, cores, imagens e muito mais, além disto, existem algumas
técnicas avançadas que permitem aos autores a construção de layouts voltados a
dispositivos móveis \cite{beginningCSSCascadingStyleSheetsForWebDesign}.

Sendo assim, o \acs{CSS} permite transformar a apresentação de um ou vários
documentos \acs{HTML}, a seguir será apresentado outro recurso que permite
dinamismo em páginas \acs{HTML}: o \acl{JS}.

\subsection{Javascript}

Segundo \citeonline{javascriptAndJQueryTheMissingManual}, \acl{JS} (\acs{JS}) é
uma linguagem de programação interpretada que permite que um documento
\acs{HTML}, que é estático, receba interatividade, animações e efeitos visuais
dinâmicos. A seguir, será apresentada uma das principais bibliotecas disponíveis
para esta linguagem: o \textit{jQuery}.

\subsubsection{Jquery}

Conforme afirma \citeonline{jQueryABibliotecaDoProgramadorJavaScript}, 
\textit{jQuery} é uma poderosa biblioteca \acl{JS} de código aberto, que foi
concebida com o objetivo de simplificar a interatividade em páginas web, seu 
criador John Resig apresentou o conceito de "\textit{write less, do more}", ou
seja, "escreva menos, faça mais", no qual em um pequeno trecho de código fonte 
utilizando a biblioteca \textit{jQuery}, é possível realizar um procedimento que
teria muito mais linhas de código, para que fosse obtido o mesmo resultado em um
\textit{script} implementado em \acs{JS} sem o uso desta biblioteca.

Por conseguinte, de acordo com \citeonline{beginningJQuery}, a biblioteca
(\textit{library}) \textit{jQuery} foi construída utilizando a linguagem de 
programação \acl{JS}, ou seja, ela não é uma nova linguagem de programação.

Portanto, a \textit{library} \textit{jQuery} tem a intenção de tornar a
programação mais fácil e divertida, sendo considerada como um programa complexo
que simplifica tarefas cotidianas com uso de seletores \acs{CSS} e resolve
problemas de compatibilidade com vários navegadores 
\cite{javascriptAndJQueryTheMissingManual}.

\subsection{Twitter Bootstrap}

Na opinião de \citeonline{jumpStartResponsiveWebDesign}, o \acs{Twitter
Bootstrap} é uma das mais famosas bibliotecas de componentes responsivos na web,
com código fonte aberto. Ele, é considerado como uma coleção de
\textit{scripts} que auxiliam na construção de aplicações web.

De maneira breve, o \textit{framework} permite que desenvolvedores possam criar
projetos web de maneira mais rápida e padronizada, além de permitir que
profissionais que não conheçam profundamente a linguagem de estilização 
\acs{CSS} desenvolvam interfaces agradáveis, pois, possui uma manual 
extensivo que aborda exemplos de implementação, além disto, permite atingir
todos os dispositivos com uso de layouts responsivos e pode ser usado com
pré-processadodores tais como: \ac{LESS} e \ac{SASS}.
