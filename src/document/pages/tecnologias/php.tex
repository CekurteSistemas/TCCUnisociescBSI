\section{PHP}

\acs{PHP} é uma linguagem de programação interpretada simples e poderosa,
projetada para criar conteúdo \acs{HTML}. Sendo que esta tecnologia pode ser
utilizada: em linha de comando, em aplicações \ac{GUI} utilizando uma biblioteca
chamada \acs{PHP-GTK} e principalmente em servidores web para gerar conteúdo
dinâmico \cite{programmingPhp}.

Sendo assim, o \acs{PHP} é um projeto \textit{open source} (de código aberto),
portanto, em outras palavras, significa que você tem acesso ao código-fonte e
pode utilizar, alterar e redistribuir tudo sem custos \cite{phpAndMysqlWebDevelopment}.

\subsection{As Versões do PHP}

A linguagem de programação \acs{PHP} surgiu em 1994, sendo que, foi criada por
Rasmus Lerdorf para o acompanhamento de visitas em seu currículo online, nesta
época foi batizada de \textit{Personal Home Page Tools}, sendo esta
frequentemente chamada apenas de \textit{PHP Tools}, no entanto, esta primeira
versão da tecnologia tratava-se apenas de um conjunto de binários \ac{CGI},
porém, em 1995 o criador da linguagem adicionou novas funcionalidades ao projeto
e o rebatizou de \ac{PHP/FI} \cite{phpProgramandoComOrientacaoAObjetos}.

Enquanto que, a segunda versão foi lançada em 1996, onde, 
Rasmus reescreveu a versão inicial e a chamou de \acs{PHP/FI} 2.0, que incluia 
suporte aos seguintes bancos de dados: \textit{mSQL},
\textit{Postgres98} e \textit{DBM} \cite{programmingPhp}.

Entretanto, devido aos problemas de instabilidade Andi Gutmans e Zeev
Suraski (fundadores da \acs{Zend}, \textbf{Ze}ev + A\textbf{nd}i) e Rasmus, em
1997 iniciaram outra reescrita da linguagem, sendo que, em 1998 foi lançada a versão
que mais se assemelha com o \acs{PHP} dos dias atuais, a versão 3, que deixou
de lado o nome \acs{PHP/FI} e passou a adotar a nomenclatura \ac{PHP}
\cite{websitePHPHistoria}.

Como descrito por \citeonline{phpProgramandoComOrientacaoAObjetos}, logo após o
lançamento da versão 3, Zeev e Andi passaram a trabalhar na reescrita do núcleo da 
linguagem afim de melhorar a sua performance, sendo assim, rebatizaram o
\textit{core} dando-lhe o nome de \textit{Zend Engine}, por conseguinte, em
2000 foi lançado o \acs{PHP} 4, que incluiu uma grande melhoria de desempenho, 
além do suporte a: sessões, vários bancos de dados e vários servidores web.

Quatro anos depois, em 2004 surge a versão 5 da linguagem \acs{PHP}, que incluí
um novo \textit{core} a \textit{Zend Engine} 2.0, incluindo dentre as principais
melhorias, o suporte à programação orientada a objetos já existente
em outras linguagens, tais como: \textit{Java} e \textit{C++}
\cite{phpProgramandoComOrientacaoAObjetos}.

Você poderá realizar o \textit{download} de qualquer uma das versão citadas
acima acessando o museu da linguagem \cite{websitePHPMuseum}.