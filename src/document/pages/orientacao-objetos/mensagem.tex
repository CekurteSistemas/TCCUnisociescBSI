\section{MENSAGEM}

Um software que foi desenvolvido utilizando os conceitos da orientação a 
objetos tem sua execução através da comunicação entre os diversos componentes 
de software, estes componentes chamados de objetos trocam mensagens com o 
objetivo de realizar uma tarefa, isto se faz necessário porque cada objeto  tem
uma responsabilidade para o qual foi projetado na fase de análise.

Assim sendo, uma mensagem é um pedido para que um objeto execute uma ação 
através da chamada de um método, sendo que, este pode alterar o estado de 
outros objetos afim de completar a sua tarefa e, quando ele finalmente termina
a sua execução, geralmente notifica quem solicitou a execução do serviço, ou
seja, retorna algum valor para o objeto que solicitou a operação 
\cite{c++ComoProgramar}.