\section{ENCAPSULAMENTO}

Até então quando definimos as propriedades para a nossa classe, deixamos elas 
com a visibilidade: privada, do inglês, \textit{private}. Sendo que, sempre que 
definirmos as propriedades e métodos em uma classe deve-se informar qual a  sua
visibilidade. Por conseguinte, existem três visibilidades, são elas: privada, 
protegida e pública. Segundo \citeonline{javaComoProgramar}, os conceitos de 
visibilidade também são chamados de modificadores de acesso.

A visibilidade privada, do inglês \textit{private}, permite acesso somente
dentro do escopo da classe que definiu a propriedade ou método. Enquanto que, a 
visibilidade protegida, do inglês \textit{protected}, permite que todas as
classes  que herdam propriedades ou métodos tenham acesso dentro do escopo da classe filha. 
Em contrapartida, a visibilidade pública, do inglês \textit{public}, permite que
qualquer objeto que tenha instanciado um objeto da classe que definiu uma propriedade ou 
método possa invoca-lo no caso do método, ou modificar o seu estado no caso de 
uma propriedade \cite{learningJava}.

Sendo assim, por questões de segurança da aplicação é interessante fornecer
interfaces para manipulação dos estados de um objeto. Por conta disto,  sempre
que definirmos uma propriedade em uma classe iremos utilizar a visibilidade 
privada, caso essa classe não implemente conceitos de herança posteriormente, 
ou então, com visibilidade protegida afim de permitir que as classes 
especialistas herdem as suas definições. E, é justamente este o conceito de
encapsulamento, realizar o ocultamento dos dados ou informações \cite{javaComoProgramar}.

Mas, para que um cliente de um objeto – todos os objetos que estão fora do
escopo da classe -  possa modificar os estados das propriedades definimos 
métodos especialistas conhecidos como: métodos \textit{getters} e
\textit{setters}.

\subsection{Métodos Getters e Setters}

São métodos responsáveis por modificar e recuperar os estados de uma variável
membro de uma classe permitindo que esta função membro esteja exposta a todo o 
escopo da aplicação.

No caso do método configurar, do inglês \textit{set}, ele permite que os
clientes de um objeto configurem novos valores para uma propriedade, realizando 
antecipadamente uma ação para validar a entrada de dados por exemplo.

Em contrapartida, existe o método obter, do inglês \textit{get}, que recupera o
valor de uma propriedade, sendo que, este método poderia realizar por exemplo uma 
formatação adequada de uma propriedade.

Sendo que, o nome desses métodos (por conta de uma convenção de atribuição de
nomes) recebe o prefixo \textit{set} ou \textit{get} seguido do nome da
propriedade, tendo esta a sua primeira letra em caixa alta \cite{javaComoProgramar}.