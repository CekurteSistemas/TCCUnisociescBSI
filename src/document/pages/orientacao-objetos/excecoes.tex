\section{EXCEÇÕES}

A linguagem de programação PHP 5, introduz o conceito de exceções, e isto traz
uma grande vantagem se comparada a manipulação de erros das versões anteriores
da tecnologia, além disto, você poderá também encontrar similaridades nos
conceitos aplicados ao PHP caso tenha conhecimento de outras linguagens, tais
como: \textit{Java} e \textit{C++} \cite{phpObjectsPatternsAndPractice}.

Uma exceção indica que ocorreu uma condição não esperada ou um erro, sendo que,
isto geralmente acontece por conta de um erro de processamento, como por
exemplo: uma atribuição ou leitura de valores incorretos ou obrigatórios em
variáveis locais e propriedades durante a execuçao de um método \cite{learningJava}.

Segundo \citeonline{phpObjectsPatternsAndPractice} uma exceção é um objeto
especial instanciado a partir da classe \textit{Exception}, ou a partir de uma
classe especialista, portanto, estes objetos são projetados para criar e
reportar informações de erro. 

Sendo assim, se comparado a forma tradicional de manipulação de erros, as
exceções são uma forma elegante de manipulá-los e tratá-los dentro de uma
aplicação, de acordo com o que vimos no capítulo \ref{heranca} que
tratava sobre herança, pode-se extender as funcionalidas da classe
\textit{Exception}, personalizando os seus dados e o seu comportamento \cite{phpMasterWriteCuttingEdgeCode}.

Conforme afirma \citeonline{phpMasterWriteCuttingEdgeCode}, um objeto da classe 
\textit{Exception}, irá conter informações referente ao erro que ocorreu, dentre
estas informações estão:

\begin{enumerate}[a)]
    \item o nome do arquivo;
    \item a linha em que ocorreu o problema;
    \item uma mensagem;
    \item e, opcionalmente, um código de erro.
\end{enumerate}

% Esta tabela está na p.52] da referencia: phpObjectsPatternsAndPractice
\begin{table}[h!tb]
	\centering
	\setlength{\belowcaptionskip}{9pt}
	\caption[Métodos públicos da classe \textit{Exception}]{\textbf{Email local X
	Email em nuvem}}
	\begin{tabular}{ l | p{0.7\textwidth} }
	
		\hline
		\textbf{Método}
		& \textbf{Descrição} \\
		\hline
		
        \textit{getMessage()} 		
        & Recupeara uma \textit{string} que foi enviada
        para o construtor.
        
        \\ \cline{1-2}
        \textit{getCode()} 			
        & Recupera o código de erro.
        \\ \cline{1-2}
        
        \textit{getFile()} 			
        & Recupera o arquivo em que a exceção foi lançada.
        \\ \cline{1-2}
        
        \textit{getLine()} 			
        & Recupera o número da linha em que a exceção ocorreu.
        \\ \cline{1-2}
        
        \textit{getPrevious()} 		
        & Recupera um objeto de uma exceção.
        \\ \cline{1-2}
        
        \textit{getTrace()} 		
        & Recupera um \textit{array} multimensional contendo as chamadas de
        métodos, incluindo: funções membro, classes, arquivos e argumentos. \\
        \cline{1-2}
        
        \textit{getTraceAsString()} 
        & Recupera os dados de \textit{getTrace()} no formato de uma
        \textit{string}.
        \\ \cline{1-2}
        
        \textit{\_\_toString()} 		
        & Método mágico executado automaticamente quando um objeto é exibido em
        tela como uma \textit{string}. Sendo assim, retorna uma \textit{string}
        descrevendo os detalhes da exceção.
        \\
        \hline
	\end{tabular}
	\newline
	\newline
	\label{tab:excecao}
	\begin{footnotesize}
		Fonte: adaptado de \cite[p.53]{phpObjectsPatternsAndPractice}
	\end{footnotesize}
\end{table}

\FloatBarrier 	% Este comando impede que as imagens 
				% flutuem a partir deste ponto no seu documento

\textit{Uma imagem que descreva como se faz para lançar uma exceção e outra
para captura-la\ldots.}