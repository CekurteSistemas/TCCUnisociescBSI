\section{EXCEÇÕES}

A linguagem de programação PHP 5, introduz o conceito de exceções, e isto traz
uma grande vantagem se comparada a manipulação de erros das versões anteriores
da tecnologia, além disto, você poderá também encontrar similaridades nos
conceitos aplicados ao PHP caso tenha conhecimento de outras linguagens, tais
como: \textit{Java} e \textit{C++} \cite{phpObjectsPatternsAndPractice}.

Uma exceção indica que ocorreu uma condição não esperada ou um erro, sendo que,
isto geralmente acontece por conta de um erro de processamento, como por
exemplo: uma atribuição ou leitura de valores incorretos ou obrigatórios em
variáveis locais e propriedades durante a execuçao de um método \cite{learningJava}.

Segundo \citeonline{phpObjectsPatternsAndPractice} uma exceção é um objeto
especial instanciado a partir da classe \textit{Exception}, ou a partir de uma
classe especialista, portanto, estes objetos são projetados para criar e
reportar informações de erro. 

Sendo assim, se comparado a forma tradicional de manipulação de erros, as
exceções são uma forma elegante de manipulá-los e tratá-los dentro de uma
aplicação, de acordo com o que vimos no capítulo \ref{heranca} que
tratava sobre herança, pode-se extender as funcionalidas da classe
\textit{Exception}, personalizando os seus dados e o seu comportamento \cite{phpMasterWriteCuttingEdgeCode}.

Conforme afirma \citeonline{phpMasterWriteCuttingEdgeCode}, um objeto da classe 
\textit{Exception}, irá conter informações referente ao erro que ocorreu, dentre
estas informações estão:

\begin{enumerate}[a)]
    \item o nome do arquivo;
    \item a linha em que ocorreu o problema;
    \item uma mensagem;
    \item e, opcionalmente, um código de erro.
\end{enumerate}

\textit{VER TABELA DE \cite[p.52]{phpObjectsPatternsAndPractice}.}

\textit{Uma imagem que descreva como se faz para lançar uma exceção e outra
para captura-la\ldots.}