\chapter{CONCLUSÃO}
\label{chp:conclusao}

Embora este projeto tenha como objetivo final entregar um sistema web de simulado
para a prova de certificação \acs{ZCPE}, esta meta não foi alcançada, pois, 
após interações de validação da ferramenta de gestão de perguntas e
respostas junto a lista de discussão, foi necessária a realização de correções e
adaptações da plataforma que, em contrapartida, deram origem aos atrasos que não
estavam previstos no cronograma de execução das atividades. Sendo assim, a
ferramenta de simulados não foi concluída.

Porém, destaca-se que o software está sendo utilizado pelos usuários da lista
de discussão para gestão de perguntas e respostas o que certamente valida a
aplicação deste estudo. Com isto, o grupo agora utiliza a ferramenta como um
repositório central e as perguntas seguem um padrão lógico graças ao template de
e-mail que foi elaborado, isto permitirá que integrações futuras possam ser
implementadas.

Percebe-se também que o trabalho ocorreu graças a união da lista de discussão,
que ofereceu apoio e aceitação durante todos os ciclos de desenvolvimento deste
projeto. Outro fato que motivou a elaboração da pesquisa supracitada é o
interesse do autor em buscar nesta certificação uma diferenciação no mercado de trabalho,
que motivado pelo seu estudo pessoal encontrou no grupo de discussão o mesmo
espírito de valorização do conhecimento presente nos profissionais que já
possuem e os que ainda buscam a certificação.

Nota-se que este é o primeiro material científico que apresenta as versões do
exame oferecidos pela \textit{ZEND}, permitindo que novos pesquisadores 
utilizem-o como referência para trabalhos futuros.

Dentre os próximos passos, sugere-se a implementação de um algoritmo de \ac{TRI}
a fim de poder analisar mais profundamente o conhecimento de um usuário durante a
realização de um simulado, o que permite avaliar se o candidato ''chutou''
alguma questão ao respondê-la. Além disto, a ferramenta proposta pode ser
perfeitamente utilizada para outras provas não só de certificação, permitindo
que usuários aproveitem-se do gerenciamento de perguntas e respostas e da 
ferramenta de simulado.