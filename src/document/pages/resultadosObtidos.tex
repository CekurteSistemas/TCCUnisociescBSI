\chapter{RESULTADOS OBTIDOS}
\label{chp:resultadosObtidos}
 
Pode-se observar na Figura XX, que a ferramenta de gestão de perguntas e
respostas está sendo utilizada para gerir as questões postadas diariamente na
lista de discussão.

IMAGEM DO GRÁFICO AQUI

Na Figura XX, destaca-se o número de visualizacoes de página na data em que a
ferramenta foi divulgada ao grupo que superior as expectativas do autor.
Percebe-se ainda que nos dias seguinte que a quantidade de visualizacoes se
estabilizou, tendo a média de XX acessos diários.

Estes acessos tiveram origem principalmente das cidades XX, XX e XX conforme
ilustra a Figura XX.

IMAGEM DA FIGURA DE ACESSOS DAS CIDADES

Nota-se ainda, que o tempo médio de acesso à ferramenta é de 02:05 minutos
conforme ilustra a Figura XX.

% Depois das imagens analiticas ..
As figuras XX, XX, XX e XX foram obtidas através de uma ferramenta analítica
denominada \textit{Google Analytics}, sendo que, os dados apresentados se
referem ao período de XX à XX.




% Agora as imagens do sistema

Na Figura XX, exibe-se a página inicial do sistema que apresenta três maneiras
de uso da plataforma.. 

IMAGEM DA TELA INICIAL DO SISTEMA

FALAR SOBRE AS 3 FORMAS DE USO

FALAR SOBRE OS NÍVEIS DE ACESSO DO SISTEMA (GRUPOS DE USUÁRIOS)

Nota-se também que o sistema proposto está internacionalizado e já conta com
dois idiomas, o português do Brasil e o inglês internacional conforme mostra a
Figura XX.

IMAGEM DA TELA INICIAL DO SISTEMA EM INGLÊS

A listagem de perguntas apresentada na Figura XX, é o resultado da categorizacao
de perguntas e respostas da lista de discussão, sendo que, nesta listagem
apresentam-se dados básicos para o usuário que está logado, e para os usuários
administradores algumas informacoes adicionais sao exibidas: são elas.. XX e XX

IMAGEM DA LISTAGEM DE PERGUNTAS E RESPOSTAS

A Figura XX, mostra-se a exibicão dos tópicos da lista de discussão diretamente
do \textit{Google Groups}, onde evidencia-se que não há somente o banco de dados
de perguntas e respostas, mas também assuntos referente a outros temas que na
Figura XX, é o caso do ``Acesso ao Dropbox''.

FALAR TAMBÉM PELA FACILIDADE DE BUSCAR UMA QUESTÃO E SABER A QUANTIDADE DE
QUESTÕES CADASTRADAS

IMAGEM DA LISTAGEM DE PERGUNTAS E RESPOSTAS DO GOOGLE GROUPS

Neste momento apresenta-se em detalhes na Figura XX, a representacão da exibicão
de detalhes de uma questão que está sendo gerenciada através da ferramenta
proposta.

IMAGEM DE DETALHES DE UMA QUESTÃO DA FERRAMENTA

Nota-se também que todas as informacões que o sistema mantém referente a questão
são exibidos, sendo que, as linhas XX e XX são exibidas somente para os usuários
com permissões administrativas.

A seguir apresenta-se na Figura XX, a mesma questão na plataforma do
\textit{Google Groups} no qual não há nenhum gerenciamento das questões que
foram postadas, diferentemente da nova forma de gestão de perguntas proposta por
este projeto.

IMAGEM DE DETALHES DE UMA QUESTÃO DA FERRAMENTA DO GOOGLE GROUPS

Como já havia sido citado, a ferramenta está sendo utilizada e está disponível
no seguinte endereco: \textit{http://zcpe.cekurte.com}.
