\chapter{INTRODUÇÃO}
\label{chp:intro}

Atualmente não há uma ferramenta gratuita que ofereça simulações de questões de 
uma prova de certificação \ac{ZCPE}. Entretanto, existe um 
grupo de discussão chamado \textit{Rumo a ZCE} onde os membros deste, postam
questões da prova, sendo que, os participantes do grupo postam as suas dúvidas 
e também respostas, e chegam em um consenso de qual seria a alternativa correta,
por conta disto, existe um banco de dados com pouco menos de 1400 perguntas 
cadastradas até o presente momento. Além disto, o ambiente em que as questões 
estão dispostas não oferece nenhum tipo de filtro, e também, não há a 
possibilidade de avaliar qual o grau de estudo dos profissionais que frequentam 
esta lista.

Então, como ajudar esses profissionais a estudarem para a prova de certificação 
de forma dinâmica aproveitando-se do banco de dados disponível no grupo de 
discussão?

\section{HIPÓTESES DE TRABALHO}

Se o profissional que almeja a certificação \acs{ZCPE} utilizar a ferramenta
realizando os simulados e, por conta disto, entender qual é o seu grau de estudo e áreas 
do conhecimento em que possuí deficiência, poderá então, focar os seus estudos 
nas áreas em que tem mais dificuldades afim de se preparar efetivamente para 
prova oficial.

\section{JUSTIFICATIVA}

Segundo \citeonline{artigoAvaliacaoSentencaOuDiagnostico} a avaliação de  alunos
faz parte do processo de desenvolvimento, sendo que, através do estabelecimento 
de critérios de avaliação é possível definir quais são os pontos fortes e  quais
os pontos fracos e que necessitam serem melhorados.

Sendo assim, hoje, através do grupo de discussão, não é possível entender em 
que área do conhecimento os profissionais possuem dificuldades, afim de focarem
os seus estudos em suas deficiências.

Por conta disto, partindo de uma ferramenta de simulado pretende-se mostrar ao
candidato após a conclusão do exame, qual a área do conhecimento em que ele 
possuí dificuldades e precisa dar atenção.

Portanto, a ferramenta servirá como apoio de
estudo para os profissionais que desejam obter a certificação \acs{ZCPE}.

\section{OBJETIVO GERAL}

Criar uma ferramenta afim de unificar as respostas do grupo de discussão
oferecendo uma interface web e permitir que o candidato realize um simulado 
de acordo com o banco de questões.

\section{OBJETIVOS ESPECÍFICOS}

\begin{enumerate}[a)]
    \item modelar a aplicação;
    \item desenvolver a aplicação com foco em usabilidade;
    \item importar perguntas e respostas do grupo de discussão;
    \item categorizar as perguntas nas áreas do conhecimento da prova de
    certificação \acs{ZCPE};
    \item avaliar o conhecimento do candidato após a conclusão de um simulado 
    de acordo com as áreas do conhecimento e expor para ele quais são as áreas  em
    que este possuí mais dificuldades;
    \item apresentar os resultados obtidos.
\end{enumerate}

\section{METODOLOGIA}

Este estudo caracteriza-se como uma pesquisa exploratória qualitativa que visa
promover o conhecimento referente ao problema de pesquisa, fazendo desta forma,
um levantamento de dados secundários bibliográficos, que para
\citeonline{comoElaborarProjetosDePesquisa} envolvem a busca do conhecimento 
teórico através de: livros técnicos e artigos.

