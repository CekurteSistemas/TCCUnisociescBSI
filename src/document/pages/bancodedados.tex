\chapter{BANCO DE DADOS}
\label{bancoDeDados}

Antes de abordarmos os dois bancos de dados livres mais populares:
\textit{MySQL} e \textit{PostgreSQL}, precisamos entender o que é um banco de
dados.

Segundo \citeonline{theDefinitiveGuideToMySQL5} um banco de dados pode ser uma
lista de registros que são manipulados por um programa de computador, como por exemplos o Excel, ou ainda,
pode ser também os arquivos de armazenamento de uma empresa de telecomunicações 
referente as várias chamadas que ocorreram diariamente, além disto, alguns
bancos de dados são utilizados por apenas um usuário, enquanto que alguns outros são 
acessados por vários usuários simultaneamente, enquanto que, uma base pode
ocupar poucos ou muitos kilobytes do dispositivo de armazenamento, sendo
assim, a palavra \textit{banco de dados} é utilizada para referenciar os
dados reais, o software gerenciador do banco (como por exemplo: \textit{MySQL} 
e \textit{PostgreSQL}), um cliente de conexão ao banco (são exemplos:  um script
PHP ou um programa escrito em C++), sendo assim, é comum que quando as  pessoas
falam sobre o assunto gerem confusão.

\section{MYSQL}

De acordo com \citeonline{integrandoPHP5ComMySQL}, \textit{MySQL} é um
\ac{RDBMS},  que utiliza a linguagem \ac{SQL} para manipular os seus registros, 
sendo que, é altamente utilizado em aplicações para a internet e, por
conseguinte,  seu código fonte é aberto, tendo destaque em características
como: velocidade, escalabilidade e confiabilidade, por conta disto, é adotado
pelo departamento de \ac{TI}, desenvolvedores e vendedores de software.

\section{POSTGRESQL}

Assim como, o \textit{MySQL}, o \textit{PostgreSQL} é um \acs{RDBMS} também de código
livre e surgiu na universidade da California, no projeto Berkeley, sendo que,
possuí alguns recursos de bancos empresariais, tais como: a abilidade de criar
funções agregadas e também a replicação de \textit{streaming}, certamente, estes
recursos raramente são encontrados em plataformas livres, mas, são comumente
encontradas em bancos comerciais, como: \textit{SQL Server} e \textit{IBM DB2}
e, além destas vantagens, ele pode superar bases comerciais em algumas cargas de
trabalho \cite{postgreSQLUpAndRunning}.
