\chapter{O QUE É A ZCE?}
\label{zce}

A \acs{ZCE}, do inglês, \acl{ZCE}, é uma prova de certificação oferecida
pela Zend que certifica que um profissional está apto a trabalhar com a
tecnologia \acs{PHP}, sendo que, esta prova era focada na versão 5.3 da
linguagem \acs{PHP}. Atualmente a organição 

esta primeira certificação é nomeada
pela instituição como \textit{Zend PHP Certification}. Entretanto, existe uma segunda prova intitulada como Zend Framework Certification, esta por sua vez, avalia os conhecimentos do profissional 
no Framework oficial da Zend \cite{websiteZendZce}. 

A Zend é uma empresa PHP que ajuda outras empresas a desenvolverem e entregarem
soluções rápidas e com alta qualidade, sejam elas: web ou mobile \cite{websiteZendCompany}.

A prova de certificação Zend foi projetada tendo como base dois objetivos, são
eles: testar o conhecimento do profissional na tecnologia PHP e, o segundo, fazer 
com que a prova extraia do profissional o máximo de sua vivência prática com a 
tecnologia \cite{theZendPHPCertificationPracticeTestBook}.

Como o escopo do PHP é muito abrangente, as grandes empresas precisam ter uma
maneira padrão e confiável de avaliar as habilidades e capacidades de um profissional 
que atue com a linguagem PHP. Sendo assim, o principal objetivo da prova é oferecer a 
empregadores e profissionais certificados uma forma de avaliação padrão
\cite{zendPhp5CertificationStudyGuide}.


Enquanto que, perante a comunidade, busca integrar os desenvolvedores PHP e
oferecer suporte para que estes profissionais criem soluções utilizando a tecnologia \cite{websiteZendCompany}.

Hoje no Brasil, segundo a \citeonline{websiteZendYellowPagesDirectory}, a quantidade de profissionais
certificados na tecnologia PHP é relativamente baixa, se comparado a quantidade 
de profissionais que trabalham com esta linguagem diariamente. Esta informação 
pode ser consultada no site da própria Zend em: \textit{Zend Certified Engineer
Directory}.

Dentre os motivos para o baixo número de profissionais certificados, acredita-se
que as principais causas sejam: o valor da prova, que perante a \citeonline{websiteZendPhpCertification}
 custa USS 195,00 e, além disto, os custos com preparações, que podem chegar até
 os USS 1000,00 no site da própria instituição \cite{websiteZendOnlineTraining}.
