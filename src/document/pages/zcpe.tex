\chapter{O QUE É A ZCPE?}
\label{zcpe}

A \acs{ZCPE}, do inglês, \acl{ZCPE}, é uma prova de certificação oferecida
pela Zend que certifica que um profissional está apto a trabalhar com a
tecnologia \acs{PHP} \cite{websiteZendZCPE}.

Entretanto, existe uma segunda prova intitulada como \ac{ZFC},
esta por sua vez, avalia os conhecimentos do profissional no
Framework oficial da Zend \cite{websiteZendZFC}.

A Zend é uma empresa que atua com \acs{PHP} e fornece a seus clientes soluções 
rápidas e com alta qualidade, sejam elas: \textit{web} ou \textit{mobile}, além
disto oferece produtos tais como o \acs{Zend Studio}, \acs{Zend Server} e 
\acs{Zend Framework} além de treinamentos e provas de certificação 
\cite{websiteZendCompany}. Enquanto que, perante a comunidade de software, busca
integrar os desenvolvedores \acs{PHP} afim de oferecer suporte para que estes 
profissionais criem soluções utilizando a tecnologia \cite{websiteZendCompany}.

Segundo \citeonline{zendPhp5CertificationStudyGuide}, com a introdução da tão
esperada certificação referente a versão 5 da linguagem \acs{PHP}, o exame se
tornou mais amplo, portanto, além do profissional ter vivência com a
tecnologia, este por sua vez, necessita de conhecimentos teóricos sólidos
referente a linguagem.

A prova de certificação Zend foi projetada tendo como base dois objetivos, são
eles: testar o conhecimento do profissional na tecnologia \acs{PHP} e, o
segundo, fazer com que a prova extraia do profissional o máximo de sua vivência 
prática com a tecnologia \cite{theZendPHPCertificationPracticeTestBook}.

Portanto, segundo \citeonline{theZendPHPCertificationPracticeTestBook}, seu
conhecimento na linguagem \acs{PHP} é baseado no seguinte princípio: sua
experiência não deve ser mensurada de acordo com tecnologias
e bibliotecas de terceiros, pois, se isto ocorresse, um profissional que
trabalha a anos com a linguagem e nunca realizou uma conexão com o banco 
\acs{MySQL} poderia ser prejudicado na avaliação, um segundo motivo é que, a
prova não deve avaliar o seu conhecimento referente a bibliotecas de software
de terceiros, desta forma, o exame aborda questões de funcionalidades da 
linguagem  de maneira didática.

Na maioria dos casos, como sempre, o exame irá avaliar a sua habilidade de
entender, interpretar e escrever códigos \acs{PHP} de maneira adequada, então,
esteja preparado para ser analizado através de exemplos de códigos nos quais
você deverá entender: como funcionam, qual a saída e se há algum \acs{bug},
além disto, algumas questões podem ser complexas, mas, se você parar para
pensar, um programador passa por isto diariamente ao analisar o código escrito
por outras pessoas para a correção de um problema
\cite{theZendPHPCertificationPracticeTestBook}.

Segundo \citeonline{theZendPHPCertificationPracticeTestBook}, o candidato que se
submete a prova de certificação \acs{ZCPE} tem noventa minutos para resolver 
setenta questões de diferentes áreas do conhecimento com dificuldades variadas.

De acordo com \citeonline{websiteZendZCE}, a prova aborda as seguintes áreas do
conhecimento:

\begin{enumerate}[a)]
    \item \acs{PHP} \textit{basics}: 				conceitos básicos referentes a
    linguagem;
    \item \textit{functions}: 						funções;
    \item \textit{data format} \& \textit{types}: 	formatos e tipos de
    dados;
    \item \textit{web features}: 					recursos web;
    \item \acs{OOP}:								\acl{OOP} ou simplemesmente \ac{POO};
    \item \textit{security}: 						segurança;
    \item \acs{I/O}: 								\acl{I/O} ou simplemesmente \ac{E/S};
    \item \textit{strings} \& \textit{patterns}: 	texto e expressões
    regulares;
    \item \textit{databases} \& \acs{SQL}: 		banco de dados e
    manipulação através de comandos \ac{SQL};
    \item \textit{arrays}: 						vetores ou matrizes.
\end{enumerate}

Perante o mercado, como o escopo do PHP é muito abrangente, as grandes empresas
precisam ter uma maneira padrão e confiável de avaliar as habilidades e 
capacidades de um profissional que atue com a linguagem PHP. Sendo assim, o
principal objetivo da prova é oferecer a empregadores e profissionais 
certificados uma forma de avaliação padrão \cite{zendPhp5CertificationStudyGuide}.

Hoje no Brasil, segundo a \citeonline{websiteZendYellowPagesDirectory}, a 
quantidade de profissionais certificados na tecnologia PHP é relativamente 
baixa, se comparado a quantidade de profissionais que trabalham com esta 
linguagem diariamente.

Dentre os motivos para o baixo número de profissionais certificados, acredita-se
que as principais causas sejam: o valor da prova, que perante a 
\citeonline{websiteZendPhpCertification} custa USD 195,00 e, além disto,  os
custos com preparações, que podem chegar até os USD 1000,00 no site da  própria
instituição \cite{websiteZendOnlineTraining}.

\section{HISTÓRICO DA PROVA}

entretanto, esta prova estava disponível no site da
instituição até dezembro de 2013, pois abordava a versão 5.3 da
linguagem \acs{PHP}, sendo que, foi substituída pela \acs{ZCPE}, do inglês, \acl{ZCPE},
que aborda a versão atual da linguagem, o PHP 5.5 
