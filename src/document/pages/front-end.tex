\chapter{DESENVOLVIMENTO FRONT-END}
\label{desenvolvimentoFrontEnd}

Geralmente quando desenvolvemos para a web, temos profissionais especialistas em
duas áreas da construção de uma aplicação, são eles: o \textit{frontend} e o
\textit{backend} \cite{artigoAvaliacaoEReducaoDoTempoDeRespostaDeSistemasWeb}.

Segundo \citeonline{artigoAvaliacaoEReducaoDoTempoDeRespostaDeSistemasWeb}, é no
\textit{frontend} que tecnologias como: a \ac{HTML}, as \ac{CSS}, o \ac{JS}
e o conteúdo multimídia são desenvolvidos.

Falaremos neste capítulo sobre essas tecnologias.

\section{HTML}
\section{CSS}

\ac{CSS}, do inglês \textit{Cascading Style Sheets}, é uma ferramenta
que webdesiners e também desenvolvedores utilizam em conjunto com a \ac{HTML} para a
construção de websites, onde, o \acs{CSS} proporciona que \ac{Browser} controle
os aspectos visuais da página, como por exemplo: o posicionamento de elementos,
estilos de texto, cores, imagens e muito mais, além disto, existem algumas
técnicas avançadas que permitem aos autores a construção de layouts voltados a
dispositivos móveis \cite{beginningCSSCascadingStyleSheetsForWebDesign}.

Sendo assim, o \acs{CSS} permite transformar a apresentação de um ou vários
documentos \acs{HTML}

\subsection{Less}
\section{JAVASCRIPT}
\subsection{Jquery}
\section{TWITTER BOOTSTRAP}

Segundo \cite{jumpStartResponsiveWebDesign}, o \acs{Twitter Bootstrap}  é uma
das mais famosas biblioteca de componentes responsivos na web.

De maneira breve, o framework permite que desenvolvedores possam criar
projetos web de maneira mais rápida e padronizada, além de permitir que
profissionais que não conheçam profundamente a linguagem de estilização 
\acs{CSS} desenvolvam interfaces agradáveis, sendo que, possuí uma manual 
extensivo que aborda exemplos de implementação, permite atingir todos os 
dispositivos com uso de layouts responsivos, pode ser usado com
pré-processadodores  tais como: \acs{LESS} e \acs{SASS} e seu código fonte é
aberto.
